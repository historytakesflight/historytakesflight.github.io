\documentclass[12pt]{article}
\usepackage[latin1]{inputenc}
\usepackage[margin=1in]{geometry}
\usepackage{amsmath}
\usepackage{amsfonts}
\usepackage{amssymb}
\usepackage{amsthm}
\usepackage{outlines}

% Start sections at 0
\setcounter{section}{-1}


% Table of Contents formatting
\usepackage{tocloft}
\renewcommand\cftsecfont{\normalfont}
\renewcommand\cftsecpagefont{\normalfont}
\renewcommand{\cftsecleader}{\cftdotfill{\cftsecdotsep}}
\renewcommand\cftsecdotsep{\cftdot}
\renewcommand\cftsubsecdotsep{\cftdot}

% formatting for links
\usepackage{hyperref}
\hypersetup{
	colorlinks=true,
	linkcolor=blue,
	filecolor=magenta,      
	urlcolor=cyan,
}

% \noindent for the entire document
\setlength\parindent{0pt} 

\begin{document}

	\begin{titlepage}
		\begin{center}
			\vspace*{1cm}
			\Huge
				\textbf{Aviation History}\\
			\vspace{0.5cm}
			\LARGE
				Independent Study Course Plan\\
			\vspace{1.5cm}
				\textbf{John Yang}\\
			\vfill
			\vspace{0.8cm}
			\Large
				Mr. Christopher Hines\\
				South Brunswick High School\\
				2020-2021
		\end{center}
	\end{titlepage}\newpage

\section{Changelog}
\textbf{November 8, 2020}
\begin{itemize}
    \item Remove ``golden age of travel" from Q1 Days 18-20; Change ``Early American Airlines"
    to ``First Airlines in America"
    \item Change Q3 Day 1 topic to ``Early Airlines" with a greater focus on airlines than aircraft
    themselves. Will discuss individual aircraft types in conjunction with their role in each
    airline.
\end{itemize}
\textbf{February 20, 2021}
\begin{itemize}
    \item Remove ``Aircraft Identification" topic; not something that's very important; more of a hobby than historically relevant. 
    \item Remove ``Modern air mail and shipping" topic; not much content here, and the main point is that most air mail is carried by cargo on passenger flights. 
    \item Adjust timeline to include extra catch-up days. 
    \item Adjust formatting and section numbering of this plan for easier readability. Add hyperlinks to TOC. 
\end{itemize}
\textbf{April 5, 2020}
\begin{itemize}
    \item Adjust timeline in Q4 to use 2 classes per incident
    \item Remove topics: TWA800; CI676; JT610 \& ET302; SV763 \& Kazakhstan 1907; JAL123 due to expanded timeframe
    \item Changed timeline to match
\end{itemize}

\newpage
\tableofcontents\newpage

\textit{This course plan is subject to change any time at the discretion of the instructor or student.}

\section*{Introduction}

In this course, the student will focus on various events in the history of aviation. Emphasis
will be put on the effects on various people around the world.

This course plan will outline specific learning activities and agendas per class period.
There are about 21-22 class periods per quarter. This plan assumes that each class period
will meet for the scheduled 88 minutes. The schedule is subject to change and will leave
some leeway in order to account for unforeseen circumstances such as fire drills, delays, etc.

Each day, the student will research the planned topics, take notes, and write webpage
articles pertaining to the topics. Notes will be organized into a GitHub repository, and
webpage articles will be continuously updated.

\section*{Summer Assignment}

The final project of this course is actually an ongoing effort. By the end of the year, the
student will have constructed a fully functional informative website detailing what they have
learned. Because of the magnitude of the project, the student will set up the backbone of
the website before the beginning of the year.

The student will use HTML and Bootstrap to construct the website layout and structure.
It will be hosted through GitHub Pages. The student will also construct an article template.
This way, the student will be able to quickly update the website during the year.

\section*{Methods of Assessment}

Throughout the course, there are planned days for assessment and review during each
quarter. At the end of each quarter will be another assessment day. During these assessments,
the student will reflect upon the progress they have made and possible improvements. This
will be expressed in a written reflection. The student will then review what they have learned
and participate in an oral discussion with the instructor. The instructor will pose questions
to the student and the student should respond in an appropriate and correct manner.

\section{Quarter 1 Plan}
\subsection{Overview}
\textbf{Focus for this quarter: Beginnings and Early Years of Aviation}

The student will cover the following topics:
\begin{outline}
\1 Early ideas of aviation
\1 18th century lighter-than-air flight
\1 Sir George Cayley and his legacy
\1 19th century developments
\1 In-depth study of the Wright Brothers and their legacy
\1 Long-distance and high-speed flight
\1 Early commercial aviation
\end{outline}

\subsection{Subtopics}
\subsubsection{Days 1-2}
\textbf{Focus Topic:} Flight from Ancient times to the 17th Century

\textbf{Research and Notes:}
\begin{outline}
    \1 Tower Jumping - Daedalus and Icarus
\1 Ancient Chinese Aircraft
\2 Kites
\2 Bamboo Helicopter
\2 Sky lanterns
\1 Early Gliders - Successes and Failures
\1 da Vinci's Ornithopter
\1 de Terzi's Airship
\1 Newton's drag equation
\end{outline}
\subsubsection{Days 3-4}
\textbf{Focus Topic:} Flight during the 18th Century - hot air balloons and lighter-than-air flight

\textbf{Research and Notes:}
\begin{outline}
    \1 Bartolomeu de Gusm\~{a}o's hot air balloon
\1 Montgolfier brothers and their hot air balloon
\1 The flight of Le Globe
\1 Manned flights from 1783-1785
\1 Jean-Pierre Blanchard's first balloon flight in the US
\1 Lavoisier's military hydrogen balloon
\1 The Company of Aeronauts in France
\end{outline}

\subsubsection{Day 5}
\textbf{Focus Topic:} Sir George Cayley, ``the father of aviation"

\textbf{Research and Notes:}
\begin{outline}
    \1 Biographical information
\1 The physics of flight
\1 Cayley's controlled glider
\1 On Aerial Navigation
\end{outline}

\subsubsection{Days 6-9}
\textbf{Focus Topic:} Flight during the 19th Century

\textbf{Research and Notes:}
\begin{outline}
    \1 The first overnight balloon flight, Nov. 7-8, 1836
\1 William Samuel Henson's drawing of a powered airplane
\1 Nadar becomes the first aerial photographer
\1 Solomon Andrews' dirigible flights
\1 Matthew Piers Watt Boulton invents the aileron
\1 1870 - the French use balloons to move letters and passengers out of Paris
\1 Wenham and Browning's wind tunnel
\1 The Zenith
\1 Enrico Forlanini's Helicopter
\1 Victor Tatin's model Airplane
\1 Gaston Tissandier makes the first electric powered flight
\1 John Joseph Montgomery's heavier-than-air controlled flying machines
\1 Cl`ement Ader's steam-powered airplane
\1 Chuhachi Ninomiya's pusher plane
\1 Otto Lilienthal
\1 Lawrence Hargrave
\1 Octave Chanute's Progress in Flying Machines
\1 Langley's Aerodrome
\1 Ernst J\"{a}gels' rigid airship flight
\1 The Zeppelin
\end{outline}

\subsubsection{Day 10}
Assessment and review; catch-up and website work
\subsubsection{Days 11-15}
\textbf{Focus Topic:} The Wright Brothers

\textbf{Research and Notes:}
\begin{outline}
    \1 Biographical information
\1 Influences \& research
\1 Controlled Flight
\1 Gliders
\1 Data Collection
\1 Powered flight and engines
\1 Skepticism
\1 Failures and Orville's accident
\1 The Patent War
\1 Flight school
\1 Death and legacy
\end{outline}
\subsubsection{Days 16-17}
\textbf{Focus Topic:} Flying longer and faster

\textbf{Research and Notes:}
\begin{outline}
    \1 The Curtiss NC-4 and crossing the Atlantic
\1 Alcock and Brown
\1 The Fokker F.IV
\1 Charles Lindbergh
\1 Flying around the world - circumnavigation
\1 Breaking the sound barrier - the Bell X-1
\end{outline}

\subsubsection{Days 18-20}
\textbf{Focus Topic:} Early civil and commercial aviation

\textbf{Research and Notes:}
\begin{outline}
    \1 Air Mail
\1 European beginnings
\1 First airlines in America
\1 The airline industry during the Great Depression
\end{outline}
\subsubsection{Day 21}
End of quarter assessment and review
\subsubsection{Day 22}
Catch-up and website work - this block may not occur in every marking period

\section{Quarter 2 Plan}
\subsection{Overview}
\textbf{Focus for this Quarter: Military Aviation}
The student will cover the following:
\begin{outline}
    \1 Early uses of aircraft
\1 Aviation during WWI and WWII
\1 Aviation during the post-war era
\1 Modern-day Military Aviation
\end{outline}
\subsection{Subtopics}
\subsubsection{Day 1}
\textbf{Focus Topic:} Use of lighter-than-air aircraft

\textbf{Research and Notes:}
\begin{outline}
    \1 The Battle of Fleurus, 1794
\1 The Civil War
\1 Zeppelins and observation balloons in WWI
\end{outline}

\subsubsection{Days 2-6}
\textbf{Focus Topic:} World War I Aviation

\textbf{Research and Notes:}
\begin{outline}
    \1 Purposes
\2 Reconnaissance
\2 Bombing
\2 Fighters
\1 Specific models and their usages
\1 Aviation in Verdun and the Somme
\end{outline}
\subsubsection{Day 7}
Catch-up and website work
\subsubsection{Days 8-15}
\textbf{Focus Topic:} World War II Aviation

\textbf{Research and Notes:}
\begin{outline}
    \1 Improvements in design and technology
\1 Anti-aircraft weapons
\1 The German air force: the Luftwaffe
\2 Dive bombing
\1 The Royal Air Force
\1 US Army Air Forces
\2 Background information
\2 Aircraft
\2 Strategies
\2 Crew rotations
\1 The Japanese
\2 Kamikaze attacks
\1 The Soviet Air Force
\1 Aviation in specific events
\2 The Battle of Britain
\2 The Battle of Normandy
\2 Pearl Harbor
\2 Hiroshima and Nagasaki
\end{outline}
\subsubsection{Day 16}
Assessment and review; Catch-up and website work
\subsubsection{Days 17-18}
\textbf{Focus Topic:} Aviation During the Post-war Era

\textbf{Research and Notes:}
\begin{outline}
    \1 The end of props and the beginning of jets
\1 Rocket engines and afterburners
\1 Rise of supersonic flight
\1 Bombers with nuclear weapons in mind
\1 Helicopters and V/STOL
\1 Digitized avionics
\end{outline}
\subsubsection{Days 19-20}
\textbf{Focus Topic:} Modern-Day Military Aviation

\textbf{Research and Notes:}
\begin{outline}
    \1 Fighters
\1 Bombers
\1 Electronic warfare
\1 Transport
\1 Unmanned aircraft and drones
\1 The Blue Angels
\1 Aircraft Carriers
\end{outline}
\subsubsection{Day 21}
End of quarter assessment and review

\subsubsection{Day 22}
Catch-up and website work day

\section{Quarter 3 Plan}
\subsection{Overview}
\textbf{Focus for this Quarter: Modern-day Civilian and General Aviation}
The student will cover the following:
\begin{outline}
    \1 Early airlines
\1 Regulators
\1 Supersonic airliners
\1 Airline Economics
\2 Competition between Airbus and Boeing
\2 Low-cost carriers
\1 Passenger experience
\1 Aircraft identification
\1 Shipping and air mail
\1 Environmental considerations
\1 General and agricultural aviation
\end{outline}
\subsection{Subtopics}
\subsubsection{Day 1}
\textbf{Focus Topic:} Early airlines

\textbf{Research and Notes:}
\begin{outline}
    \1 Aircraft:
\2 Douglas DC-3
\2 The de Havilland Comet
\2 Boeing 707
\2 Tupolev 104
\2 DC-8, VC10, Il-62
\2 The Boeing 747
\2 Advantages of jet aircraft
\1 Airlines:
\2 US legacy airlines and previous constituents
\end{outline}
\subsubsection{Day 2}
\textbf{Focus Topic:} Civil aviation authorities

\textbf{Research and Notes:}
\begin{outline}
    \1 The Chicago Convention and ICAO
\1 IATA
\1 The FAA
\1 Other global regulators
\end{outline}
\subsubsection{Day 3}
\textbf{Focus Topic:} The rise and fall of supersonic transport

\textbf{Research and Notes:}
\begin{outline}
    \1 The Concorde
\1 Tupolev 144
\1 Why did they fail?
\end{outline}
\subsubsection{Day 4}
\textbf{Focus Topic:} Airbus enters the civil aviation game

\textbf{Research and Notes:}
\begin{outline}
    \1 Foundations
\1 A300
\1 Competition with Boeing
\1 The Airbus A380 - successes and failures
\1 A220/CS300 Fiasco
\end{outline}

\subsubsection{Days 5-8}
\textbf{Focus Topic:} Airline Economics

\textbf{Research and Notes:}
\begin{outline}
    \1 Terminology
\1 Demand
\1 Pricing and Scheduling
\1 Metrics
\1 Frequent flyer programs, points, and miles
\end{outline}

\subsubsection{Days 9-10}
Catch-up days: Articles 

\subsubsection{Day 11}
Assessment and review; Catch-up and website work day
\subsubsection{Days 12-13}
\textbf{Focus Topic:} The rise of low-cost carriers

\textbf{Research and Notes:}
\begin{outline}
    \1 Southwest Airlines' strategy
\1 General business model
\1 Auxiliary revenue
\1 IATA LCC characteristics
\1 Differences (e.g. Ryanair vs. JetBlue)
\1 ULCCs
\1 New aircraft and LCCs
\end{outline}

\subsubsection{Days 14-15}
\textbf{Focus Topic:} Modern-day passenger experience

\textbf{Research and Notes:}
\begin{outline}
    \1 Degradation of domestic US premium cabin products
\1 Economy class on a legacy carrier
\1 International airlines vs. US airlines
\1 Middle-eastern premium cabins
\1 Air cuisine
\1 Domestic regional travel
\1 Transcontinental travel
\end{outline}
\subsubsection{Day 16}
\textbf{Focus Topic:} Environmental considerations of the airline industry

\textbf{Research and Notes:}
\begin{outline}
    \1 Repositioning flights
\1 Fuel consumption
\2 United Airlines and biofuel
\1 Pollution concerns
\1 Electric/solar aircraft
\1 ``Chemtrails"
\end{outline}
\subsubsection{Day 17}
Catch-up day 
\subsubsection{Day 18}
\textbf{Focus Topic:} General aviation

\textbf{Research and Notes:}
\begin{outline}
    \1 Flight training
\1 Small piston aircraft (C172, PA-28, SR22)
\1 GA airports and FBOs
\1 The AOPA
\end{outline}
\subsubsection{Days 19-20}
\textbf{Focus Topic:} Death of airlines

\textbf{Research and Notes:}
\begin{outline}
    \1 Economic factors
\1 Mergers
\1 Bankruptcies
\1 Specific examples
\2 Thomas Cook
\2 Air Berlin
\end{outline}
\subsubsection{Day 21}
End of quarter assessment and review
\subsubsection{Day 22}
Catch-up and website work

\section{Quarter 4 Plan}
\subsection{Overview}
\textbf{Focus for this Quarter: Aviation Safety and Incidents}
This quarter will be divided into two parts. First, the student will learn about current and
historical safety and security measures in the aviation industry. Topics include:
\begin{outline}
    \1 Airport security
\1 Safety hazards
\end{outline}
Second, the student will read official accident reports by regulatory bodies, and summarize
the events in website articles. The student should consider the following factors:
\begin{outline}
    \1 Background information and contributing factors
\1 Mistakes made and the cause of the accident
\1 Action to take in order to avoid similar situations
\end{outline}
\subsection{Part 1 Subtopics}
\subsubsection{Day 1}
\textbf{Focus Topic:} Airport security

\textbf{Research and Notes:}
\begin{outline}
    \1 TSA and its effectiveness
\1 ICAO airport security standards
\1 Israeli Airport Security
\end{outline}
\subsubsection{Days 2-3}
\textbf{Focus Topic:} Hazards

\textbf{Research and Notes:}
\begin{outline}
    \1 Miscommunication/misinformation
\1 FOD
\1 Weather-related
\2 Ice/snow
\2 Winds
\2 VFR into IMC
\1 Engine failure
\1 Structural issues
\1 Stalls/Loss of control
\1 Fire/smoke
\1 Human factors
\end{outline}
\subsubsection{Day 4}
Assessment and review; catch-up and website work
\subsection{Part 2 Subtopics}
\subsubsection{Day 5}
\textbf{Incident:} The Hindenburg Disaster
\subsubsection{Days 6-8}
\textbf{Incident:} 9/11
\begin{outline}
    \1 Timeline of events
\1 AA11, UA175, AA77, UA93
\1 Aftermath
\1 Airport security response
\1 Airline industry response
\end{outline}
\subsubsection{Day 9}
Catch-up and website work
\subsubsection{Days 10-11}
\textbf{Incidents:} DC 10 issues: UA232, AA96, TK981, AA191
\subsubsection{Days 12-13}
\textbf{Incident:} Tenerife disaster
\subsubsection{Days 14-15}
\textbf{Incident:} Pan Am 103
\subsubsection{Days 16-17}
\textbf{Incident:} AA587
\subsubsection{Days 18-19}
\textbf{Incident:} MH370

% \subsubsection{Day 14}
% \textbf{Incident:} TWA 800
% \subsubsection{Day 15} 
% \textbf{Incident:} CI676
% \subsubsection{Day 16}
% \textbf{Incident:} JT610, ET302
% \subsubsection{Day 18}
% \textbf{Incident:} JAL123

\subsubsection{Day 20}
Catch-up and website work days
\subsubsection{Days 21-22}
End of year review/reflection:
\begin{outline}
    \1 Oral assessment
\1 Reflection essay
\1 Website evaluation
\end{outline}

\end{document}
